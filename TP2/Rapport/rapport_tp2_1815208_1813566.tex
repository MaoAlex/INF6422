%%%%%%%%%%%%%%%%%%%%%%%%%%%%%%%%%%%%%%%%%%{{{
% Structured General Purpose Assignment
% LaTeX Template
%
% This template has been downloaded from:
% http://www.latextemplates.com
%
% Original author:
% Ted Pavlic (http://www.tedpavlic.com)
%
% Note:
% The \lipsum[#] commands throughout this template generate dummy text
% to fill the template out. These commands should all be removed when 
% writing assignment content.
%
%%%%%%%%%%%%%%%%%%%%%%%%%%%%%%%%%%%%%%%%%

%----------------------------------------------------------------------------------------
%	PACKAGES AND OTHER DOCUMENT CONFIGURATIONS
%----------------------------------------------------------------------------------------

\documentclass[12pt,a4paper]{article}

\usepackage{fancyhdr} % Required for custom headers
\usepackage[utf8]{inputenc}
\usepackage[french]{babel}
\usepackage{hyperref}
\usepackage{hyperref}
\usepackage{float}
\usepackage{lastpage} % Required to determine the last page for the footer
\usepackage{extramarks} % Required for headers and footers
\usepackage{graphicx} % Required to insert images
\usepackage{lipsum} % Used for inserting dummy 'Lorem ipsum' text into the template

% Margins
\topmargin=-0.45in
\evensidemargin=0in
\oddsidemargin=0in
\textwidth=6.5in
\textheight=9.0in
\headsep=0.25in 

\linespread{1.1} % Line spacing

% Set up the header and footer
\pagestyle{fancy}
\lhead{\hmwkTitle} % Top left header
\rhead{\firstxmark} % Top right header
\lfoot{\lastxmark} % Bottom left footer
\cfoot{} % Bottom center footer
\rfoot{Page\ \thepage\ de\ \pageref{LastPage}} % Bottom right footer
\renewcommand\headrulewidth{0.4pt} % Size of the header rule

\renewcommand\footrulewidth{0.4pt} % Size of the footer rule

\setlength\parindent{0pt} % Removes all indentation from paragraphs

%----------------------------------------------------------------------------------------
%	DOCUMENT STRUCTURE COMMANDS
%	Skip this unless you know what you're doing
%----------------------------------------------------------------------------------------

% Header and footer for when a page split occurs within a problem environment
\newcommand{\enterProblemHeader}[1]{
\nobreak\extramarks{#1}{#1 continue \ldots}\nobreak
\nobreak\extramarks{#1 (suite)}{#1 continue \ldots}\nobreak
}

% Header and footer for when a page split occurs between problem environments
\newcommand{\exitProblemHeader}[1]{
\nobreak\extramarks{#1 (continued)}{#1 continued on next page\ldots}\nobreak
\nobreak\extramarks{#1}{}\nobreak
}

\setcounter{secnumdepth}{0} % Removes default section numbers
\newcounter{homeworkProblemCounter} % Creates a counter to keep track of the number of problems
\newcounter{homeworkQuestionCounter} % Creates a counter to keep track of the number of problems

\newcommand{\homeworkProblemName}{}
\newenvironment{homeworkProblem}[1][Problem \arabic{homeworkProblemCounter}]{ % Makes a new environment called homeworkProblem which takes 1 argument (custom name) but the default is "Problem #"
\stepcounter{homeworkProblemCounter} % Increase counter for number of problems
\renewcommand{\homeworkProblemName}{#1} % Assign \homeworkProblemName the name of the problem
\section{\homeworkProblemName} % Make a section in the document with the custom problem count
\enterProblemHeader{\homeworkProblemName} % Header and footer within the environment
}{
\exitProblemHeader{\homeworkProblemName} % Header and footer after the environment
\setcounter{homeworkQuestionCounter}{0} % Removes default section numbers
}

\newcommand{\problemAnswer}[1]{ % Defines the problem answer command with the content as the only argument
\noindent\framebox[\columnwidth][c]{\begin{minipage}{0.98\columnwidth}#1\end{minipage}} % Makes the box around the problem answer and puts the content inside
}

\newcommand{\homeworkSectionName}{}
\newenvironment{homeworkSection}[1]{ % New environment for sections within homework problems, takes 1 argument - the name of the section
\stepcounter{homeworkQuestionCounter} % Increase counter for number of problems
\renewcommand{\homeworkSectionName}{\arabic{homeworkProblemCounter} . \arabic{homeworkQuestionCounter} #1} % Assign \homeworkSectionName to the name of the section from the environment argument
\subsection{\homeworkSectionName} % Make a subsection with the custom name of the subsection
\enterProblemHeader{\homeworkProblemName\ [\homeworkSectionName]} % Header and footer within the environment
}{
\enterProblemHeader{\homeworkProblemName} % Header and footer after the environment
}
   
%----------------------------------------------------------------------------------------
%	NAME AND CLASS SECTION
%----------------------------------------------------------------------------------------

\newcommand{\hmwkTitle}{TP2} % Assignment title
\newcommand{\hmwkDueDate}{Mardi\ 16 Février\ 2016} % Due date
\newcommand{\hmwkClass}{INF6422} % Course/class
\newcommand{\hmwkClassTime}{12h} % Class/lecture time
\newcommand{\hmwkClassInstructor}{François Labrèche} % Teacher/lecturer
\newcommand{\hmwkAuthorName}{Philippe Troclet (1815208) et Alexandre Mao (1813566)} % Your name

%----------------------------------------------------------------------------------------
%	TITLE PAGE
%----------------------------------------------------------------------------------------

\title{
\vspace{2in}
\textmd{\textbf{\hmwkClass:\ \hmwkTitle}}\\
\normalsize\vspace{0.1in}\small{pour\ le\ \hmwkDueDate}\\
\vspace{3in}
}

\author{\textbf{\hmwkAuthorName}}
\date{} % Insert date here if you want it to appear below your name

%----------------------------------------------------------------------------------------

\begin{document}

\maketitle

%----------------------------------------------------------------------------------------
%	TABLE OF CONTENTS
%----------------------------------------------------------------------------------------

%\setcounter{tocdepth}{1} % Uncomment this line if you don't want subsections listed in the ToC

\newpage
\tableofcontents
\newpage%}}}

%Pour mettre des images
%\begin{figure}[H]
    %\begin{center}
	%\includegraphics{Images/} 
        %\caption{Légende}
        %\label{fig:reference}
    %\end{center}
%\end{figure}

%----------------------------------------------------------------------------------------
%	Première partie
%----------------------------------------------------------------------------------------

\begin{homeworkProblem}[\arabic{homeworkProblemCounter} Méthode statistique] % Custom section title

\begin{homeworkSection}{Régression logitique} % Section within problem
L'autre 33\% des données va servira à tester le modèle conçus à partir de la phase d'apprentisage, c'est donc l'échantillon test sur lequel on connaît on va vérifier la véracité de notre modèle.
(tableau avec les valeur des variables indépendantes).
Le coefficient correspond à ...
Son "odd ratio" correspond à la corrélation entre la probabilité que le courriel soit un spam dans la population d'apprentissage par rapport à la probabilité que celui-ci le soit dans la population que l'on teste. C'est une mesure du facteur de spam.


\end{homeworkSection}

%--------------------------------------------

\begin{homeworkSection}{Etude du Modèle} % Section within problem

Pour notre modèle de régression linéaire, on a :
\begin{itemize}
	\item Un taux de vrai positif de 0.891 pour les spams et un taux de vrai positif de 0.954 pour les mails non-spam (la moyenne étant de 0.929), le taux de vrai positif. Il représente les taux de mails qui ont été considéré comme étant des spams et qui sont réellement des mails 
	\item un taux de faux positif de 0.046 qui est le taux de mail considérés comme des spams mais à tort. Et un taux de faux positif de 0.109 pour les mails considérés comme des non-spams alors qu'ils le sont
	\item une précision de (0.926 pour les spams/0.931 pour les mails normaux) 0.929 de moyenne.
La précision correspond au rapport entre le nombre de mails détecté comme des spams et qui le sont effectivement, sur le nombre total de mail qui sont des spams pour la première valeur. Pour la seconde c'est la même chose mais avec les mails qui ne sont pas des spams.
La moyenne se fait donc sur le nombre de mails qui ont été bien catégorisé sur le nombre total de mail.
	\item une sensibilité (recall) de (0.891/0.954) 0.929. On remarque que la sensibilité a les mêmes valeurs que les true positives
	\item un F-measure de (0.908/0.942) 0.929. Ces valeurs ont l'air d'être la moyenne entre la précision et la sensibilité.
	\item une aire sous la courbe (ROC area) de 0.971. Cette valeur mesure la performance de notre modèle
%A voir https://fr.wikipedia.org/wiki/Receiver_Operating_Characteristic
	\item On a pour matrice de confusion : {{550, 67}, {44, 903}}, avec la première ligne a = 1, première colonne a, deuxième ligne b=2 et colonne b. Cette matrice représente le nombre de machines catégorisées comme suivantes : spam étant spam, spam étant non spam, courriel ordinaire étant courriel ordinaire et  courriel ordinaire étant spam.
\end{itemize}

\end{homeworkSection}

\begin{homeworkSection}{Tokenization attack} 
Un exemple de contre-mesure de type Tokenization attack qu'un spammeur pourrait facilement utiliser afin de contourner un filtre basé uniquement sur la fréquence d'apparition de certains mots est de permuter la place de certaines lettres dans certains mots, ou de créer de manière volontaire des fautes d'orthographes qui n'altéreront pas la signification du texte.
Pour le message suivant nous pourrions avoir le résultat :

DEAR RECEIVER,

You have just recieved a Talban virus. Since we are not so technlogoicaly advanced in Afganistan, this is a MANUAL vrus. Please click on this lnk(http://clickeme.com) to delte all the files on your hard disk and send this mal to everyone you know. 
%voir article http://nlp.stanford.edu/IR-book/html/htmledition/tokenization-1.html
\end{homeworkSection}
%--------------------------------------------

\end{homeworkProblem}

%----------------------------------------------------------------------------------------
%	PROBLEM 3
%----------------------------------------------------------------------------------------

\begin{homeworkProblem}[\arabic{homeworkProblemCounter} Apprentissage automatique] % Roman numerals
Parmi les méthodes d'apprentissage automatique les plus courantes, nous avons :
\begin{itemize}
\item l'apprentissage non-supervisé ("clustering") est un apprentissage automatique. Il s'agit pour un logiciel de diviser un groupe hétérogène de données, en sous-groupes de manière que les données considérées comme les plus similaires soient associées au sein d'un groupe homogène et qu'au contraire les données considérées comme différentes se retrouvent dans d'autres groupes distincts ; l'objectif étant de permettre une extraction de connaissance organisée à partir de ces données %définition de wikipédia https://fr.wikipedia.org/wiki/Apprentissage_non_supervis%C3%A9
\item L'apprentissage semi-supervisé est un apprentissage.
L'apprentissage semi-supervisé est une classe de techniques d'apprentissage automatique qui utilise un ensemble de données étiquetées et non-étiquetés. Il se situe ainsi entre l'apprentissage supervisé qui n'utilise que des données étiquetées et l'apprentissage non-supervisé qui n'utilise que des données non-étiquetées. Il a été démontré que l'utilisation de données non-étiquetées, en combinaison avec des données étiquetées, permet d'améliorer significativement la qualité de l'apprentissage. Un autre intérêt provient du fait que l'étiquetage de données nécessite l'intervention d'un utilisateur humain. Lorsque les jeux de données deviennent très grands, cette opération peut s'avérer fastidieuse. Dans ce cas, l'apprentissage semi-supervisé, qui ne nécessite que quelques étiquettes, revêt un intérêt pratique évident. %wikipédia https://fr.wikipedia.org/wiki/Apprentissage_semi-supervis%C3%A9
\item L'apprentissage supervisé ("supervised learning")est une technique d'apprentissage automatique où l'on cherche à produire automatiquement des règles à partir d'une base de données d'apprentissage contenant des « exemples » (en général des cas déjà traités et validés). %wikipédia https://fr.wikipedia.org/wiki/Apprentissage_supervis%C3%A9 
\end{itemize}
%--------------------------------------------

\begin{homeworkSection}{Caractéristiques des méthodes} % Using the problem name elsewhere
\end{homeworkSection}

%--------------------------------------------

\begin{homeworkSection}{Classification naïve bayésienne}
Pour notre modèle de classification naïve bayésienne, on a :
\begin{itemize}
	\item Un taux de vrai positif de 0.945 pour les spams et un taux de vrai positif de 0.673 pour les mails non-spam (la moyenne étant de 0.78), le taux de vrai positif. Il représente les taux de mails qui ont été considéré comme étant des spams et qui sont réellement des mails 
	\item un taux de faux positif de 0.327 qui est le taux de mail considérés comme des spams mais à tort. Et un taux de faux positif de 0.0055 pour les mails considérés comme des non-spams alors qu'ils le sont. moyenne de 0.163
	\item une précision de (0.653 pour les spams/0.949 pour les mails normaux) 0.832 de moyenne.
La précision correspond au rapport entre le nombre de mails détecté comme des spams et qui le sont effectivement, sur le nombre total de mail qui sont des spams pour la première valeur. Pour la seconde c'est la même chose mais avec les mails qui ne sont pas des spams.
La moyenne se fait donc sur le nombre de mails qui ont été bien catégorisé sur le nombre total de mail.
	\item une sensibilité (recall) de (0.945/0.673) 0.78. On remarque que la sensibilité a les mêmes valeurs que les true positives
	\item un F-measure de (0.772/0.787) 0.781. Ces valeurs ont l'air d'être la moyenne entre la précision et la sensibilité. %définition à revoir
	\item une aire sous la courbe (ROC area) de 0.935. Cette valeur mesure la performance de notre modèle
	\item On a pour matrice de confusion : {{583, 34}, {310, 637}}, avec la première ligne a = 1, première colonne a, deuxième ligne b=2 et colonne b.
%A voir https://fr.wikipedia.org/wiki/Receiver_Operating_Characteristic
Comparativement au précédent modèle, ce que nous pouvons voir, c'est qu'au niveau des moyenne, ce modèle à des résultats moins bon que le modèle avec la régression linéaire malgré le fait qu'il détecte mieux les spams, il détecte en revanche un part très importante des mails normaux comme des spams (1/3 environ), ce qui n'est pas une bonne chose du tout et pourrait être préjudiciable aux utilisateurs
\end{itemize}
\end{homeworkSection}

\begin{homeworkSection}{Les arbres décisionnels}
Pour notre modèle de random forest, on a :
\begin{itemize}
	\item Un taux de vrai positif de 0.927 pour les spams et un taux de vrai positif de 0.963 pour les mails non-spam (la moyenne étant de 0.949), le taux de vrai positif. Il représente les taux de mails qui ont été considéré comme étant des spams et qui sont réellement des mails 
	\item un taux de faux positif de 0.037 qui est le taux de mail considérés comme des spams mais à tort. Et un taux de faux positif de 0.073 pour les mails considérés comme des non-spams alors qu'ils le sont. moyenne de 0.059
	\item une précision de (0.927 pour les spams/0.963 pour les mails normaux) 0.949 de moyenne.
La précision correspond au rapport entre le nombre de mails détecté comme des spams et qui le sont effectivement, sur le nombre total de mail qui sont des spams pour la première valeur. Pour la seconde c'est la même chose mais avec les mails qui ne sont pas des spams.
La moyenne se fait donc sur le nombre de mails qui ont été bien catégorisé sur le nombre total de mail.
	\item une sensibilité (recall) de (0.927/0.963) 0.949. On remarque que la sensibilité a les mêmes valeurs que les true positives
	\item un F-measure de (0.935/0.958) 0.949. Ces valeurs ont l'air d'être la moyenne entre la précision et la sensibilité. %définition à revoir
	\item une aire sous la courbe (ROC area) de 0.987. Cette valeur mesure la performance de notre modèle
	\item On a pour matrice de confusion : {{572, 45}, {35, 912}}, avec la première ligne a = 1, première colonne a, deuxième ligne b=2 et colonne b.
%A voir https://fr.wikipedia.org/wiki/Receiver_Operating_Characteristic
Comparativement aux deux précédents modèles, ce que nous pouvons voir, est que les valeurs pour les différents élément sont toutes meilleures. Nous pouvons donc en déduire que ce modèle est préférable
\end{itemize}
\end{homeworkSection}

\begin{homeworkSection}{Contre-mesures \textit{Satistical}}
Un exemple de contre-mesure de type Statistical attack qu'un spammeur pourrait utiliser afin d'échapper à un filtre basé sur la fréquence des mots en utilisant une méthode d'apprentissage automatique est l'utilisation de synonymes pour les mots, cela abaisserait la fréquence des mots qui sont considérés comme très présent dans les spams. % A compléter
Une solution qui permettrait de contrecarrer cette contre-mesure est : ....
\end{homeworkSection}
\end{homeworkProblem}

%--------------------------------------------

\begin{homeworkProblem}[\arabic{homeworkProblemCounter} Apprentissage automatique] % Roman numerals

\begin{homeworkSection}{Comparaison des résultats}
Au niveau des résultats, nous pouvons voir que la régression linéaire nous fournit un modèle correcte, le modèle de classification naïve bayésienne nous fournit un modèle non pertinent car il considère plus de 30\% des mails comme spam alors qu'ils ne le sont pas. Et le modèle de forêts d'arbres décisionnels nous donne le meilleur résultat.
C'est donc cette méhodes qui semblent donner les meilleures performances pour le jeu de données spambase
\end{homeworkSection}

\begin{homeworkSection}{Avantage et inconvénient d'un modèle non supervisé}
\end{homeworkSection}

\begin{homeworkSection}{Amélioration de la performance}
\end{homeworkSection}


\end{homeworkProblem}

\end{document}
