%%%%%%%%%%%%%%%%%%%%%%%%%%%%%%%%%%%%%%%%%%{{{
% Structured General Purpose Assignment
% LaTeX Template
%
% This template has been downloaded from:
% http://www.latextemplates.com
%
% Original author:
% Ted Pavlic (http://www.tedpavlic.com)
%
% Note:
% The \lipsum[#] commands throughout this template generate dummy text
% to fill the template out. These commands should all be removed when 
% writing assignment content.
%
%%%%%%%%%%%%%%%%%%%%%%%%%%%%%%%%%%%%%%%%%

%----------------------------------------------------------------------------------------
%	PACKAGES AND OTHER DOCUMENT CONFIGURATIONS
%----------------------------------------------------------------------------------------

\documentclass[12pt,a4paper]{article}

\usepackage{fancyhdr} % Required for custom headers
\usepackage[utf8]{inputenc}
\usepackage[french]{babel}
\usepackage{hyperref}
\usepackage{lastpage} % Required to determine the last page for the footer
\usepackage{extramarks} % Required for headers and footers
\usepackage{graphicx} % Required to insert images
\usepackage{lipsum} % Used for inserting dummy 'Lorem ipsum' text into the template

% Margins
\topmargin=-0.45in
\evensidemargin=0in
\oddsidemargin=0in
\textwidth=6.5in
\textheight=9.0in
\headsep=0.25in 

\linespread{1.1} % Line spacing

% Set up the header and footer
\pagestyle{fancy}
\lhead{\hmwkAuthorName \ \hmwkTitle} % Top left header
\rhead{\firstxmark} % Top right header
\lfoot{\lastxmark} % Bottom left footer
\cfoot{} % Bottom center footer
\rfoot{Page\ \thepage\ of\ \pageref{LastPage}} % Bottom right footer
\renewcommand\headrulewidth{0.4pt} % Size of the header rule
\renewcommand\footrulewidth{0.4pt} % Size of the footer rule

\setlength\parindent{0pt} % Removes all indentation from paragraphs

%----------------------------------------------------------------------------------------
%	DOCUMENT STRUCTURE COMMANDS
%	Skip this unless you know what you're doing
%----------------------------------------------------------------------------------------

% Header and footer for when a page split occurs within a problem environment
\newcommand{\enterProblemHeader}[1]{
\nobreak\extramarks{#1}{#1 continued on next page\ldots}\nobreak
\nobreak\extramarks{#1 (continued)}{#1 continued on next page\ldots}\nobreak
}

% Header and footer for when a page split occurs between problem environments
\newcommand{\exitProblemHeader}[1]{
\nobreak\extramarks{#1 (continued)}{#1 continued on next page\ldots}\nobreak
\nobreak\extramarks{#1}{}\nobreak
}

\setcounter{secnumdepth}{0} % Removes default section numbers
\newcounter{homeworkProblemCounter} % Creates a counter to keep track of the number of problems
\newcounter{homeworkQuestionCounter} % Creates a counter to keep track of the number of problems

\newcommand{\homeworkProblemName}{}
\newenvironment{homeworkProblem}[1][Problem \arabic{homeworkProblemCounter}]{ % Makes a new environment called homeworkProblem which takes 1 argument (custom name) but the default is "Problem #"
\stepcounter{homeworkProblemCounter} % Increase counter for number of problems
\renewcommand{\homeworkProblemName}{#1} % Assign \homeworkProblemName the name of the problem
\section{\homeworkProblemName} % Make a section in the document with the custom problem count
\enterProblemHeader{\homeworkProblemName} % Header and footer within the environment
}{
\exitProblemHeader{\homeworkProblemName} % Header and footer after the environment
\setcounter{homeworkQuestionCounter}{0} % Removes default section numbers
}

\newcommand{\problemAnswer}[1]{ % Defines the problem answer command with the content as the only argument
\noindent\framebox[\columnwidth][c]{\begin{minipage}{0.98\columnwidth}#1\end{minipage}} % Makes the box around the problem answer and puts the content inside
}

\newcommand{\homeworkSectionName}{}
\newenvironment{homeworkSection}[1]{ % New environment for sections within homework problems, takes 1 argument - the name of the section
\stepcounter{homeworkQuestionCounter} % Increase counter for number of problems
\renewcommand{\homeworkSectionName}{\arabic{homeworkProblemCounter} . \arabic{homeworkQuestionCounter} #1} % Assign \homeworkSectionName to the name of the section from the environment argument
\subsection{\homeworkSectionName} % Make a subsection with the custom name of the subsection
\enterProblemHeader{\homeworkProblemName\ [\homeworkSectionName]} % Header and footer within the environment
}{
\enterProblemHeader{\homeworkProblemName} % Header and footer after the environment
}
   
%----------------------------------------------------------------------------------------
%	NAME AND CLASS SECTION
%----------------------------------------------------------------------------------------

\newcommand{\hmwkTitle}{TP1} % Assignment title
\newcommand{\hmwkDueDate}{Mardi\ 2 Février\ 2016} % Due date
\newcommand{\hmwkClass}{INF6422} % Course/class
\newcommand{\hmwkClassTime}{12h} % Class/lecture time
\newcommand{\hmwkClassInstructor}{François Labrèche} % Teacher/lecturer
\newcommand{\hmwkAuthorName}{Philippe Troclet (1815208) et Alexandre Mao (1813566)} % Your name

%----------------------------------------------------------------------------------------
%	TITLE PAGE
%----------------------------------------------------------------------------------------

\title{
\vspace{2in}
\textmd{\textbf{\hmwkClass:\ \hmwkTitle}}\\
\normalsize\vspace{0.1in}\small{pour\ le\ \hmwkDueDate}\\
\vspace{3in}
}

\author{\textbf{\hmwkAuthorName}}
\date{} % Insert date here if you want it to appear below your name

%----------------------------------------------------------------------------------------

\begin{document}

\maketitle

%----------------------------------------------------------------------------------------
%	TABLE OF CONTENTS
%----------------------------------------------------------------------------------------

%\setcounter{tocdepth}{1} % Uncomment this line if you don't want subsections listed in the ToC

\newpage
\tableofcontents
\newpage%}}}

%----------------------------------------------------------------------------------------
%	Première partie
%----------------------------------------------------------------------------------------

\begin{homeworkProblem}[\arabic{homeworkProblemCounter} Modèle déterministe] % Custom section title
\lipsum[3] % Question

%--------------------------------------------

\begin{homeworkSection}{Choix d'un modèle comportemental} % Section within problem
    Dans la question précédente, nous avions déterminé qu'un modèle compartiental de type SI s'appliquait à l'étude de la
propagation des logiciels malveillants dans un réseau. Sachant qu'un ordinateur peut être sain ou infecté, et uniquement l'un de
ces deux états, on a la première relation:
\[
    S + I = N
\]
Où $N$ est la taille de la population et $S$ la taille de la population saine. Tandis que $I$ est la population infectée. Si de
plus, on note $\lambda$ le nombre de contacts par machine par unité de temps, On a alors que \( \lambda \cdot I \) représente le
nombre de machine qui ont été atteintes par une machine infecté entre deux unités de temps. Sachant que \( \frac{S}{N} \) est la
proportion de machines saines à l'instant courant, on peut approximer le nombre de machines nouvellement infectées entre deux pas
de temps par \( \lambda \cdot I \cdot \frac{S}{N} \). Ce qui nous donne l'équation différentielle suivante:
\[
    \frac{dI(t)}{dt} = \lambda \cdot I(t) \cdot \frac{S(t)}{N(t)}
\]
On peut alors remplacer $S(t)$ par $N(t) - I(t)$, et on obtient:
\[
    \frac{dI(t)}{dt} = \lambda \cdot I(t) \cdot \frac{N(t)-I(t)}{N(t)}
\]
\end{homeworkSection}

%--------------------------------------------

\begin{homeworkSection}{Identification des équations différentielles} % Section within problem
\problemAnswer{ % Answer
\lipsum[6]
}
\end{homeworkSection}

%--------------------------------------------

\end{homeworkProblem}

%----------------------------------------------------------------------------------------
%	PROBLEM 3
%----------------------------------------------------------------------------------------

\begin{homeworkProblem}[\arabic{homeworkProblemCounter} Simulation numériques] % Roman numerals

%--------------------------------------------

\begin{homeworkSection}{Etude de I et S en fonction du temps} % Using the problem name elsewhere
\problemAnswer{ % Answer
\lipsum[7]
}
\end{homeworkSection}

%--------------------------------------------

\begin{homeworkSection}{Etude de l'influence du paramètre $\lambda$}
\lipsum[8]\vspace{10pt} % Question

\problemAnswer{ % Answer
\lipsum[9]
}
\end{homeworkSection}

%--------------------------------------------

\end{homeworkProblem}

%----------------------------------------------------------------------------------------
%	PROBLEM 4
%----------------------------------------------------------------------------------------

\begin{homeworkProblem}[\arabic{homeworkProblemCounter} Modèle stochastique] % Roman numerals
\problemAnswer{ % Answer
\lipsum[10]
}
\end{homeworkProblem}

%----------------------------------------------------------------------------------------

\end{document}
